\documentclass[12pt, a4paper]{article}

\usepackage{xeCJK}          % 处理中文核心宏包
\setCJKmainfont{SimSun}     % 设置中文字体为宋体 (请根据系统字体调整,如 FandolSong)
\usepackage{amsmath}        % 数学公式支持
\usepackage{amssymb}        % 数学符号
\usepackage{geometry}       % 页面边距设置
\usepackage{booktabs}       % 专业表格线 (toprule, midrule, bottomrule)
\usepackage{longtable}      % 支持跨页的长表格
\usepackage{array}          % 表格列格式控制
\usepackage{hyperref}       % 超链接支持
\usepackage{xltabular} 
\geometry{left=2.5cm, right=2.5cm, top=1.5cm, bottom=1.8cm}

\title{题目: 高频数据量化因子分布式计算}
\author{}
\date{}

\begin{document}

\maketitle

行情快照数据记录了每个时间现价订单簿的10档价量数据,数据为 tradeTime 3秒频数据。

\begin{longtable}{p{3.5cm} p{3.5cm} p{2.5cm} p{5.5cm}}
\caption{行情快照数据字段说明} \\
\toprule
\textbf{字段名称} & \textbf{字段英文名} & \textbf{类型} & \textbf{说明} \\
\midrule
\endfirsthead

\toprule
\textbf{字段名称} & \textbf{字段英文名} & \textbf{类型} & \textbf{说明} \\
\midrule
\endhead

\bottomrule
\endfoot

交易日期 & tradingDay & Int32 & 交易日期,如 yyyymmdd \\
交易时间 & tradeTime & Int64 & 交易时间,如 hmmssffffffff \\
接收时间 & recvtime & Int64 & 委托接受时间,如 hmmssffffffff \\
市场识别码 & MIC & Object & XSHE,深交所, XSHG,上交所, XBEI,北交所 \\
股票代码 & code & Object & \\
累计成交笔数 & cumCut & Int32 & 当天开盘以来的成交笔数总和 \\
累计成交量 & cumVol & Int64 & 当天总成交股数 \\
累计成交金额 & turnover & Int64 & 当天总成交金额 \\
最新成交价 & last & Int64 & 最新一笔成交价格 \\
开盘价 & open & Int64 & 当天第一笔成交的价格 \\
最高价 & high & Int64 & 当日迄今为止最高成交价 \\
最低价 & low & Int64 & 当日迄今为止最低成交价 \\
全市场买单总量 & tBidVol & Int64 & 当前全部买单的总股数 \\
全市场卖单总量 & tAskVol & Int64 & 当前全部卖单的总股数 \\
加权平均买价 & wBidPrc & Int64 & 以买单数量加权平均的买价 \\
加权平均卖价 & wAskPrc & Int64 & 以卖单数量加权平均的卖价 \\
持仓量 & openInterest & Int64 & 通常用于期货/可转债市场,表示未平仓合约数;对股票而言无意义,因此为0或int64最小值 \\
买一价 & bp1 & Int64 & 买方最优价格 \\
买一量 & bv1 & Int64 & 买一价位的委托数量 \\
卖一价 & ap1 & Int64 & 卖家最优报价 \\
卖一量 & av1 & Int64 & 卖一价位的委托数量 \\
买二价 & bp2 & Int64 & \\
买二量 & bv2 & Int64 & \\
卖二价 & ap2 & Int64 & \\
卖二量 & av2 & Int64 & \\
买三价 & bp3 & Int64 & \\
买三量 & bv3 & Int64 & \\
卖三价 & ap3 & Int64 & \\
卖三量 & av3 & Int64 & \\
买四价 & bp4 & Int64 & \\
买四量 & bv4 & Int64 & \\
卖四价 & ap4 & Int64 & \\
卖四量 & av4 & Int64 & \\
买五价 & bp5 & Int64 & \\
买五量 & bv5 & Int64 & \\
卖五价 & ap5 & Int64 & \\
卖五量 & av5 & Int64 & \\
买六价 & bp6 & Int64 & \\
买六量 & bv6 & Int64 & \\
卖六价 & ap6 & Int64 & \\
卖六量 & av6 & Int64 & \\
买七价 & bp7 & Int64 & \\
买七量 & bv7 & Int64 & \\
卖七价 & ap7 & Int64 & \\
卖七量 & av7 & Int64 & \\
买八价 & bp8 & Int64 & \\
买八量 & bv8 & Int64 & \\
卖八价 & ap8 & Int64 & \\
卖八量 & av8 & Int64 & \\
买九价 & bp9 & Int64 & \\
买九量 & bv9 & Int64 & \\
卖九价 & ap9 & Int64 & \\
卖九量 & av9 & Int64 & \\
买十价 & bp10 & Int64 & \\
买十量 & bv10 & Int64 & \\
卖十价 & ap10 & Int64 & \\
卖十量 & av10 & Int64 & \\
\end{longtable}



\newpage

\newcommand{\smallerCJK}[1]{{\CJKfontspec{SimSun}\fontsize{9}{11}\selectfont #1}}
\newcommand{\smallerfont}{\small\CJKfontspec{SimSun}}
\renewcommand{\tabularxcolumn}[1]{m{#1}}

\begingroup
\smallerfont
\renewcommand{\arraystretch}{1.3} 
\setlength{\extrarowheight}{8pt}
\everymath{\displaystyle}

\begin{xltabular}{\linewidth}{|c|l|l|>{\vspace{7pt}}X<{\vspace{7pt}}|}
\caption{20个因子详细定义} \\
\hline
\smallerCJK{\textbf{序号}} & \smallerCJK{\textbf{因子名称}} & \smallerCJK{\textbf{简要描述}} & \multicolumn{1}{c|}{\smallerCJK{\textbf{公式 (以第 $t$ 时刻为准)}}} \\ 
\hline
\endfirsthead

\hline
\smallerCJK{\textbf{序号}} & \smallerCJK{\textbf{因子名称}} & \smallerCJK{\textbf{简要描述}} & \multicolumn{1}{c|}{\smallerCJK{\textbf{公式 (以第 $t$ 时刻为准)}}} \\
\hline
\endhead

\hline
\endfoot

1 & \smallerCJK{最优价差} & \smallerCJK{买一卖一价差} & $ap1_{t}-bp1_{t}$ \\
\hline
2 & \smallerCJK{相对价差} & \smallerCJK{价差相对于中间价的比例} & $\dfrac{ap1_{t}-bp1_{t}}{(ap1_{t}+bp1_{t})/2}$ \\
\hline
3 & \smallerCJK{中间价} & \smallerCJK{买卖一档均价} & $(ap1_{t}+bp1_{t})/2$ \\
\hline
4 & \smallerCJK{买一不平衡} & \smallerCJK{买卖一档挂单不平衡} & $\dfrac{bv1_{t}-av1_{t}}{bv1_{t}+av1_{t}}$ \\
\hline
5 & \smallerCJK{前5档多档不平衡} & \smallerCJK{买卖量不平衡} & $\dfrac{\sum_{i=1}^{5}bv(i)_{t}-\sum_{i=1}^{5}av(i)_{t}}{\sum_{i=1}^{5}bv(i)_{t}+\sum_{i=1}^{5}av(i)_{t}}$ \\
\hline
6 & \smallerCJK{前5档买方深度} & \smallerCJK{买单总量} & $\sum_{i=1}^{5}bv(i)_{t}$ \\
\hline
7 & \smallerCJK{前5档卖方深度} & \smallerCJK{卖单总量} & $\sum_{i=1}^{5}av(i)_{t}$ \\
\hline
8 & \smallerCJK{买卖深度差} & \smallerCJK{深度差} & $\sum_{i=1}^{5}bv(i)_{t}-\sum_{i=1}^{5}av(i)_{t}$ \\
\hline
9 & \smallerCJK{买卖深度比} & \smallerCJK{深度比值} & $\dfrac{\sum_{i=1}^{5}bv(i)_{t}}{\sum_{i=1}^{5}av(i)_{t}}$ \\
\hline
10 & \smallerCJK{全市场买卖量平衡指数} & \smallerCJK{买卖总量平衡指标} & $\dfrac{tBidVol_{t}-tAskVol_{t}}{tBidVol_{t}+tAskVol_{t}}$ \\
\hline
11 & \smallerCJK{前5档买方加权价格} & \smallerCJK{买价按挂单量加权平均} & $\dfrac{\sum_{i=1}^{5}(bp(i)_{t}\cdot bv(i)_{t})}{\sum_{i=1}^{5}bv(i)_{t}}$ \\
\hline
12 & \smallerCJK{前5档卖方加权价格} & \smallerCJK{卖价按挂单量加权平均} & $\dfrac{\sum_{i=1}^{5}(ap(i)_{t}\cdot av(i)_{t})}{\sum_{i=1}^{5}av(i)_{t}}$ \\
\hline
13 & \smallerCJK{综合加权中价} & \smallerCJK{加权中间价} & $\dfrac{\sum_{i=1}^{5}bp(i)_{t}\cdot bv(i)_{t}+\sum_{i=1}^{5}ap(i)_{t}\cdot av(i)_{t}}{\sum_{i=1}^{5}bv(i)_{t}+\sum_{i=1}^{5}av(i)_{t}}$ \\
\hline
14 & \smallerCJK{买卖加权价差} & \smallerCJK{加权价差(参考因子11,12)} & $\text{VWAPAsk}\_t(5) - \text{VWAPBid}\_t(5)$ \\
\hline
15 & \smallerCJK{每档平均挂单量差} & \smallerCJK{买卖密度差} & $\dfrac{1}{5}\sum_{i=1}^{5}bv(i)_{t}-\dfrac{1}{5}\sum_{i=1}^{5}av(i)_{t}$ \\
\hline
16 & \smallerCJK{买卖不对称度} & \smallerCJK{按档位衰减加权的不平衡} & $\dfrac{\sum_{i=1}^{5}\dfrac{1}{i}{bv(i)_{t}}-\sum_{i=1}^{5}\dfrac{1}{i}av(i)_{t}}{\sum_{i=1}^{5}\dfrac{1}{i}{bv(i)_{t}}+\sum_{i=1}^{5}\dfrac{1}{i}{av(i)_{t}}}$ \\
\hline
17 & \smallerCJK{最优价变动} & \smallerCJK{最优报价变化幅度} & $ap1_{t}-ap1_{t-1}$ \\
\hline
18 & \smallerCJK{中间价变动} & \smallerCJK{中间价的变化} & $\dfrac{1}{2}[(ap1_{t}+bp1_{t})-(ap1_{t-1}+bp1_{t-1})]$ \\
\hline
19 & \smallerCJK{深度比变动} & \smallerCJK{买卖深度比的变化率} & $\dfrac{\sum_{i=1}^{5}bv(i)_{t}}{\sum_{i=1}^{5}av(i)_{t}}-\dfrac{\sum_{i=1}^{5}bv(i)_{t-1}}{\sum_{i=1}^{5}av(i)_{t-1}}$ \\
\hline
20 & \smallerCJK{价压指标} & \smallerCJK{价差相对于深度的压力指标} & $\dfrac{ap1_{t}-bp1_{t}}{\sum_{i=1}^{5}bv(i)_{t}+\sum_{i=1}^{5}av(i)_{t}}$ \\
\hline
\end{xltabular}
\endgroup

\noindent \textbf{参数:} $n=5 \quad \Delta t=1$










% =========================================================
% 第四部分:输出数据
% =========================================================
\newpage

\section{输出数据}

\subsection*{训练阶段(各小组自行编写 MapReduce 阶段):}
给定原始数据集(20240102-20240108 沪深300指数全部股票数据),程序需要计算得到第三部分给出的所有因子的因子值序列。对于每一天数据,输出所有因子在300只股票上从9:30:00 开始到15:00:00的平均因子值序列。输出格式要求参考标准输出,第一行为列名,分别是 tradeTime 和因子代码(alpha\_n),从第二行开始输出时刻与每个因子在300只股票上的平均因子值。

$f_{t}(i)$ 资产在时间t的因子值,在本 project 中资产指的是CSI300 股票, $i=1,...,N_{t}$ 当期资产池内的资产, $N_{t}=300$:
\begin{equation}
\tilde{f}_{t}=\frac{1}{N_{t}}\sum_{i=1}^{N_{t}}f_{t}(i)
\end{equation}

\subsection*{测试阶段(第16周现场测试):}
给定另一天的沪深300指数全部股票数据,要求输出所有因子在300只股票上从9:30:00开始到 15:00:00的平均因子值序列。输出格式要求与上面一致。

\vspace{1em}
\noindent *因子值计算过程可能出现分母为0的情况,在分母上添加极小值 $1e^{-7}$。

% =========================================================
% 第五部分:原始数据下载
% =========================================================
\section{原始数据}

\textbf{校内网下载:}
深沪两市 20240102-20240108共五天的行情快照数据表,校内坚果云数据下载链接:
\begin{itemize}
    \item csi300.zip - 坚果云南方科技大学
    \item 因子值标准答案- 坚果云南方科技大学
\end{itemize}
\textbf{提示:}行情快照数据表数据中,不是所有字段都是对于本任务有用的。

% =========================================================
% 第六部分:评分标准
% =========================================================
\section{评分标准}

整个 project 占总评的40\%,即满分40分。

其中,
\begin{itemize}
    \item $\checkmark$ \textbf{技术报告(reports)占5分},包含问题描述,任务理解,难点分析,整体技术方案(图和文字详细描述),代码的模块化设计思路等。代码必须有详细注释,与有效代码行数相比,至少达到1:1比例。主要考察文档撰写的清晰程度。
    \item $\checkmark$ \textbf{展示(presentation)占5分}。包括对任务的理解,整体技术方案,代码的模块化设计思路等。
    \item $\checkmark$ \textbf{代码(codes)占30分},包括准确性测试,速度测试。准确性占20分(一题一分),速度占10分(按时间排名来算)。
\end{itemize}

必须使用HDFS+MapReduce 的方式设计方案和编程实现,必须使用JAVA语言。我们给出因子值作为标准答案供大家完成 project。测试时将采用另外的数据进行测试。

\subsection*{准确性测试}

按最终输出的数据的正确性进行评分。我们给出若干股票作为示例数据及标准答案供大家完成 project。测试时将采用另外的数据进行测试并基于新数据进行准确性评分。对于每一题,同学计算出的因子平均值与标准因子平均值的误差不超过1\%,误差计算方式为 (标准平均值- 平均值),则算正确,该题得1分;否则该题不得分。

\subsubsection*{标准平均值}
行业通用的评估指标采用 Pearson IC,计算公式如下。本次项目不涉及,仅供自行了解。

$f_{t}(i)$: 资产在时间t的因子值,在本 project 中资产指的是CSI300 股票。

$R_{t\rightarrow t+h}(i)$: 资产i从t到 $t+h$ 的未来收益,为 $(\frac{P_{(t+h)}(i)}{P_{t}(i)}-1)\%$ , $h=10\hat{r}$ 时间步。

$P_{t}(i)$: 资产在t时刻的最新成交价格(last)。

\begin{itemize}
    \item $i=1,...,N_{t}:$ 当期资产池内的资产
    \item $\tilde{f}_{t}=\frac{1}{N_{t}}\sum_{i=1}^{N_{t}}f_{t}(i)$ , $\overline{R}_{t}=\frac{1}{N_{t}}\sum_{i=1}^{N_{t}}R_{t\rightarrow t+h}(i)$
\end{itemize}

\textbf{Pearson IC(横截面线性相关):}
\begin{equation}
IC_{t}=\frac{\sum_{i=1}^{N_{t}}(f_{t}(i)-\overline{f}_{t})(R_{t\rightarrow t+h}(i)-\overline{R}_{t})}{\sqrt{\sum_{i=1}^{N_{t}}(f_{t}(i)-\overline{f}_{t})^{2}}\sqrt{\sum_{t=1}^{N_{t}}(R_{t\rightarrow t+h}(i)-\overline{R}_{t})^{2}}}
\end{equation}

\subsection*{速度测试:}

按整体程序在课程分配的 docker 中的运行时间来评分。具体评分机制按照运行时间排序分组,时间排序前5\%的组获得此项所有分数,时间排序后5\%的组获得此项40\%分数,分数按照5\%分组从最快到最慢线性递减。

\end{document}
